\documentclass[a4paper, 12pt]{article}
\usepackage[top=2cm, bottom=2cm, left=2.5cm, right=2.5cm]{geometry}
\usepackage[utf8]{inputec}
\begin{document}
Formatação Básica de Texto
    \begin{center}
    Centralizar

    \textbf{Equação polinomial do 2° grau.}

    \end{center}

    \begin{flushright}
    Para a esquerda
        
    \textit{Equação polinomial do 2° grau.}
    \end{flushright}

    \begin{flushleft}
    Para a direita
        
    \underline{Equação polinomial do 2° grau.}
    \end{flushleft}

    \begin{center}
    Centralizar e aplicar as 3 formatações
        
    \textbf{\textit{\underline{Equação polinomial do 2° grau.}}}
        
    \end{center}

Uma equação da forma $$ax^2 + bx + c = 0,$$ $a \neq 0$ será chamada de equação
polinomial do 2° grau.

A solução dessa equação é dada por
$$x = \frac{-b \pm \sqrt{b^2 - 4ac}}{2ac} $$

\end{document}