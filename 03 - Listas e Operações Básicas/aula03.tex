\documentclass[a4paper, 12pt]{article}
\usepackage[top=2cm, bottom=2cm, left=2.5cm, right=2.5cm]{geometry}
\usepackage[utf8]{inputec}
\usepackage{amsmath, amsfonts, amssymb}
\begin{document}
Listas e Operações Básicas
\section{Numerar em sequência, 1. 2. 3. ....}
    \begin{enumerate}
        itens
        \item mais outro texto.
        \begin{enumerate}
            subitens
            \item{aqui vem o texto}
            \item{aqui vem o texto}
                \begin{enumerate}
                    \item{outro texto}
                \end{enumerate}
        \end{enumerate}
        \item vem o texto.
        \item vem o texto.
        \item outro texto.
    \end{enumerate}


\begin{itemize}
    bolinha
    \item isso aqui é um item.
    \item isso aqui é outro item.
        \begin{itemize}
            \item não sei oque.
            \item ainda não sei.
            \item sei lá.
        \end{itemize}
    \item mais outro item.


\section{Operações}
$a + b$

$a - b$

$a\cdot b$

$a \times b$

$a \div b$

isso é uma fração $\frac{a}{b}$

uma linha de tsta para ver como fica

isso é uma fração $\dfrac{a}{b}$

\section{Raízes e Potências}

$\sqrt{a}{b}$

$\sqrt[3]{a}$

$\sqrt[8]{a}$

$a^b$

$a^b+c$

$a^{b+c}$

$a^10$

$a^{10}$

\section{Índices}

$a_1$, $a_2$, $a_3$ e $a_4$

$a_{10}$, $a_{20}$



\end{document}