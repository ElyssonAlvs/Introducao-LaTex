\documentclass[a4paper, 12pt]{article}
\usepackage[top=2cm, bottom=2cm, left=2.5cm, right=2.5cm]{geometry}
\usepackage[utf8]{inputec}
\usepackage{amsmath, amsfonts, amssymb}
\begin{document}
Notações de Conjunto
    \begin{enumerate}
        \item Sejam os conjuntos $A = \{1;\, 2;\, 3;\, 4\}$,
        $B = \{x \in \mathbb{Z} \, |\, -2 \leq x < 4\}$ e
        $C = \{x \in \mathbb{N} \, |\, x \geq 2\}$. Responda aos itens abaixo.
        \begin{enumerate}
            \item $A \cap B$ intersessão de conjuntos
            \item $B \cup C$ união de conjuntos
            \item $A - C$ difrerença de connjutos (1)
            \item $C \setminus B$ diferença de conjuntos (2)
        \end{enumerate}
    \end{enumerate}
\textbf{Outras Notações}
    \begin{enumerate}
        \item Classifique em verdadeiro ou falso.
        \begin{enumerate}
            \item $\mathbb{Z} \subset \mathbb{N}$ Inteiro contido nos Naturais.
            \item $\mathbb{R} \supset \mathbb{Q}$ Reais contém os Racionais.
            \item $\mathbb{Z} \not\subset \mathbb{N}$ Inteiro não está contido nos Naturais.
            \item $\mathbb{R} \not\supset \mathbb{Q}$ Reais não contém os Racionais.
            \item $0 \not\in \mathbb{R} \setminus\mathbb{Q}$ 0 não pertence aos Reais menos Racionais.
            \item $\forall z \in \mathbb{N}$, temos $x \geq 0$ Para todo x pertencente aos Naturais, temos x maior ou igual a 0.
            \item $\exists x \in \mathbb{R}$, tal que $\sqrt{x} \not\in\mathbb{R}$ Existe um x pertecente aos Reais, tal que a raiz quadrada de x não pertence ao Reais.
            \item $7 \not\in \{x\in \mathbb{N} \,|\, x \textrm{ é par}\}$
            \item $-5 \in \mathbb{R}^*_+$ 5 pertence aos Reais positivos
            \item $0 \in \varnothing$ 0 pertence ao vazio
        \end{enumerate}
    \end{enumerate}

\end{document}